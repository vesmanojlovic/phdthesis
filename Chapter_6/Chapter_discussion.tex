\chapter{Discussion}\label{chapter:discussion}

\section{Summary}
Modelling cancer evolution is difficult. Like with any data-driven approach,
straying too far into general laws may lead to underfitting of patient-specific
data. Focussing too much on the individual, however, makes overfitting the main
issue. As novel sequencing techniques, or at least public data sets which use
them, have become more available, there is an increased emphasis on
multi-pronged approaches \cite{heide_co-evolution_2022,
househam_phenotypic_2022}. In my view, a similar conundrum extends one level of
abstraction above this, as focussing on data too much may lead to missing the
proverbial forest. Much like some branches of mathematics grew out of necessity
to quantify physical phenomena, so has mathematical theory informed experiments
decades after development. I believe a similar approach is essential in
mathematical oncology as well, with robust mathematical frameworks translating
into the laboratory or even clinical trials. A good example would be adaptive
therapy, which has the potential to revolutionise patient care and is based on
optimising drug scheduling \cite{west_survey_2023}.\par

In chapter \ref{chapter:trees}, I investigated properties of the universal tree
balance index $J^1$. Originally introduced to analyse cancer phylogenies, I
show that the roots of the idea for such a metric go surprisingly deep, with
pioneering works in computer science defining an effectively identical formula.
The expected value of the index under even the simplest tree generation
processes was too complex to calculate analytically, so I relied on the
relationship derived in the original $J^1$ paper \cite{lemant_robust_2022}
which connects it with the Sackin index to obtain an accurate approximation.
Another challenge came from deriving tree families which minimise the index.
While I proved that there exists a family of trees on which the value of $J^1$
is lower than on the caterpillar tree, traditionally the tree that minimises
balance indices, I could only conjecture that this family indeed minimises the
index across all trees on a given number of leaves. \par

Balance is only one property of many trees posses. This meant that the next
logical step was to consider the values of a set of indices during a tree
generating process. Inspired by \cite{noble_spatial_2022}, I used a modified set
of evolutionary indices on the outputs of an agent-based model to test how well
one can distinguish different evolutionary trajectories in the space described
by the indices. These results I compared to a new set of tree shape indices
\cite{noble_new_2023}, which generalise $J^1$ further on trees with defined
branch lengths. While these indices have not produced different results from the
smaller set, the fact that they still show a clear separation between different
evolutionary modes is promising. \par

In chapter \ref{chapter:methdemon}, I narrowed down my consideration of the
modes of tumour evolution to the specific case of effectively neutral evolution,
which manifests itself via progressive differentiation. The model I developed in
this chapter restricts the effects of selection to patches of cells, with the
patches themselves not interacting or interacting neutrally. I demonstrated that
ABC could be employed to recover parameters of the model with varying degrees of
accuracy. \par

Having established a workflow inspired by a concrete data set, in chapter
\ref{chapter:methylation} I recovered aspects of evolutionary dynamics of
colorectal cancer using its methylation array data. Building on prior work on
fluctuating methylation clocks, I modelled multi-site bulk sequences of tumour
glands. By considering the differences between spatially close and distant
glands, as well as individual gland fCpG arrays, I was able to approximate the
epimutation and gland fission rates relative to cancer stem cell division rates.
By inferring the rate of tumour growth, it may be possible to more accurately
estimate the age of individual tumours in future studies.

\section{Hybrid modelling and inference}
ABM is a powerful tool for studying the evolutionary dynamics of cancer, but the
stochastic nature of small-scale events can make it difficult to obtain results
which are closely aligned with the data, and \texttt{methdemon} is no exception.
Prior work in fCpG modelling was done using a likelihood-based approach
\cite{gabbutt_fluctuating_2022, gabbutt_evolutionary_2023}, and has shown
promising results. Due to the complexity of colorectal cancer, an exclusively
likelihood-based approach may not be feasible, but a hybrid model which
leverages likelihood-based inference locally for detecting potentially small
effects of selection, with ABC on the global scale could be a good compromise.
The main hurdle in this approach is reconciling fissions with the steady-state
turnover process. An approach I am currently exploring is combining tau-leaping
for intra-gland dynamics and Gillespie's algorithm for gland fissions.

\section{Gland phylogenies}
Another piece of the colon cancer fCpG puzzle is the reconstruction of gland
phylogenetic trees. In \cite{gabbutt_evolutionary_2023}, the authors used a
custom BEAST pipeline, a Bayesian phylogenetic inference tool
\cite{bouckaert_beast_2019}, to infer clone phylogenies from blood cancer data.
The main differences between the two data sets include the fact that the blood
cancer data is non-spatial, and therefore possible for meaningful sequencing at
multiple points in time. In the case of colorectal cancer, it is not possible
to obtain multiple samples from the same gland over time. Further, the spatial
nature of the data means that sequencing it in the first place may not be
possible before the tumour has been removed. This means that the clock rate of
the gland phylogenies is unknown. However, sequencing multiple glands does
allow for accurate reconstruction of tree topologies, with the clock rate being
a nuisance parameter. Having discussed the potential for tree shape indices
being used in evolutionary mode inference, the next logical step would be
testing whether they point to signs of global selective pressures in the data.
Because of the way \texttt{methdemon} is written, phylogenies are easily
constructed as a byproduct of the simulation, making it easy to test
effectively neutral growth as a null model. Resources permitting, it would be
informative to use a larger-scale spatial model to test different ways of gland
organisation in space, which could result in different modes of evolution, and
thus tree shapes.

\section{Conclusion}
In this thesis, I have taken a multi-faceted approach to modelling cancer
evolution. Chapters \ref{chapter:trees} and \ref{chapter:trajectories} go down
the path of tree shape indices, which are emerging as a powerful tool for
analysing driver phylogenies, as well as trees more generally. Condensing the
information contained in a tree into an array of indices allows for a meaningful
comparison of trees, which could easily be integrated in popular statistical
packages for tree inference. On the other hand, chapters \ref{chapter:methdemon}
and \ref{chapter:methylation} explore evolution of colorectal cancer using
agent-based modelling and approximate Bayesian computation. There is a rich
literature on the topic of colorectal cancer ranging from tumourigenesis to
metastasis, but this is the first time that the evolutionary dynamics of the
disease have been modelled based on fluctuating methylation clocks. As discussed
in chapter \ref{chapter:methylation}, the model is based on a certain set of
assumptions, which may at times be too restrictive. Despite this, the model
has been capable of recovering evolutionary properties of the disease, and seems
to reliably estimate how fast the tumour has grown. \par
The research presented in this thesis, based on mathematical and computational
modelling of evolutionary dynamics of cancer, contributes to the better
understanding of the ways in which cancer can evolve. From differentiating
evolutionary trajectories to leveraging the mode of evolution of a tumour in
developing a model to describe it, the work presented here offers ample scope
for applying and extending these methods in future work.

