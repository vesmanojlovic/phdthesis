%This is the part of the preface in which the user can define his/her functions.

%% Custom enumerate environment:
\newenvironment{packed_enum}{
\begin{enumerate}[a)]
  \setlength{\itemsep}{1pt}
  \setlength{\parskip}{0pt}
  \setlength{\parsep}{0pt}
}{\end{enumerate}}


%%Imagen 2 x root locus:
\newcommand{\tablefigs}[8]{
\begin{figure}[#8]
\centering
\begin{tabular}{cc}
\hspace{-4mm}
\subfloat[#3]{\includegraphics[width=0.5\linewidth]{#1/#2}\label{#2}}\hspace{0mm}
\subfloat[#5]{\includegraphics[width=0.5\linewidth]{#1/#4}\label{#3}}\hspace{0mm}
\end{tabular}
\caption{#6}
\label{#7}
\end{figure}}


%%%%%%%% ----------- ROTATION MATRICES ----------- %%%%%%%%

%%%%----------------- MATRIZ ROT Z -----------------%%%%
\newcommand{\zrot}[1]{
%\begin{displaymath}
\left(
\begin{array}{ccc}
\cos#1 & -\sin#1 & 0 \\
\sin#1 &  \cos#1 & 0 \\
	0 	& 	  0    & 1 \\
\end{array}
\right)
%\end{displaymath}
}
%%%%------------------------------------------------%%%%

%%%%----------------- INVERS ROT Z -----------------%%%%
\newcommand{\zroti}[1]{
%\begin{displaymath}
\left(
\begin{array}{ccc}
 \cos#1 & \sin#1 & 0 \\
-\sin#1 & \cos#1 & 0 \\
	0 	 & 	  0    & 1 \\
\end{array}
\right)
%\end{displaymath}
}
%%%%------------------------------------------------%%%%




%%%%----------------- MATRIZ ROT Y -----------------%%%%
\newcommand{\yrot}[1]{
%\begin{displaymath}
\left(
\begin{array}{ccc}
 \cos#1 & 0 & \sin#1	\\
	0	 & 1 &	0 		\\
-\sin#1 & 0 & \cos#1	\\
\end{array}
\right)
%\end{displaymath}
}
%%%%------------------------------------------------%%%%

%%%%----------------- INVERS ROT Y -----------------%%%%
\newcommand{\yroti}[1]{
%\begin{displaymath}
\left(
\begin{array}{ccc}
\cos#1 & 0 & -\sin#1	\\
	0	& 1 &	0 		\\
\sin#1 & 0 &  \cos#1	\\
\end{array}
\right)
%\end{displaymath}
}
%%%%------------------------------------------------%%%%




%%%%----------------- MATRIZ ROT X -----------------%%%%
\newcommand{\xrot}[1]{
%\begin{displaymath}
\left(
\begin{array}{ccc}
1 &	   0	&	  0 	\\
0 &	\cos#1 & -\sin#1	\\
0 &	\sin#1 &  \cos#1	\\
\end{array}
\right)
%\end{displaymath}
}
%%%%------------------------------------------------%%%%

%%%%----------------- INVERS ROT X -----------------%%%%
\newcommand{\xroti}[1]{
%\begin{displaymath}
\left(
\begin{array}{ccc}
1 &	    0	 &	  0 	\\
0 &	 \cos#1 & \sin#1	\\
0 &	-\sin#1 & \cos#1	\\
\end{array}
\right)
%\end{displaymath}
}
%%%%------------------------------------------------%%%%

\newcommand\numberthis{\addtocounter{equation}{1}\tag{\theequation}}
\newcommand\underrel[2]{\mathrel{\mathop{#2}\limits_{#1}}}
\theoremstyle{plain}
\newtheorem{theorem}{Theorem}[section]
\newtheorem{proposition}{Proposition}[section]
\newtheorem{conjecture}{Conjecture}[section]


\theoremstyle{definition}
\newtheorem{definition}{Definition}[section]

\theoremstyle{remark}
\newtheorem{remark}{Remark}[section]
\newtheorem{corollary}{Corollary}[section]

