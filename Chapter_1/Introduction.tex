\chapter{Introduction}\label{chapter:introduction}
TO DO:
\begin{itemize}
    \item Introduction - general mathematical oncology overview
    \item Intro subsection - phylogenetic trees generally and in cancer evo
    \item Intro subsection - agent-based models in cancer evo
    \item Intro subsection - fCpGs for lineage tracing in cancer
    \item Intro subsection - statistical methods (ABC)
\end{itemize}

Cancer remains one of the most formidable challenges in the realm of health and medicine, causing a quarter of all deaths in the UK \cite{cruk_2022}.
Despite advances in cancer research, the survival rates for many cancers remain low, with the disease being an increasing burden on healthcare
systems \cite{financial_burden}. The disease's heterogeneity, both within and between patients, is a major obstacle to effective treatment. Understanding the
underlying evolutionary processes driving this heterogeneity is crucial to developing new treatments and improving patient outcomes.
While having a comprehensive mathematical theory of cancer evolution may not be feasible, concrete mathematical models can provide valuable insights into the
disease's dynamics. To this end, we consider different approaches to modelling cancer evolution via driver mutations, which includes the use of phylogenetic
trees and agent-based models. Further, we employ methylation data to verify the accuracy of our models using Approximate Bayesian Computation (ABC).\par
Trees as a mathematical object have found use in a variety of fields, of which biology is our main focus. However, while writing this thesis, we have found
treesinteresting links to methods in computer science via information theory. This will is discussed in more detail in section \ref{section:trees} and chapter
\ref{chapter:trees}.

\section{Trees and their applications}
In the most general sense, a tree is a connected graph with no cycles. In this thesis, when a tree is
mentioned, we refer to a rooted tree, as formally defined in section \ref{section:preliminary_defs}.
Trees have found use in a variety of fields, including computer science, biology, and linguistics.
In computer science, trees are used to represent hierarchical data structures, such as file systems and
the structure of web pages. The concept of search trees, dating back to the mid 20th century,
revolutionised the field of computer science with applications in information retrieval in the form of
binary search trees and self-balancing trees (cite knuth, nievergelt, and all that good stuff). In biology,
trees date back to the 19th century, when Charles Darwin used them to represent the evolutionary
relationships between species. Phylogenetic trees have over time become a key tool in analysing the
lineages of species, viral mutations, and cancer evolution. However, due to the different approach to
trees in these two fields, the terminology diverges quickly. Furthermore, in computer science, the
various properties of trees are quantified in certain metrics, such as the entropy and balance of a tree.
In evolutionary biology, these tools have spent years in development hell, with proprietary approaches
developed for different applications, rather than using a common framework. In fact, there are at least
$19$ different metrics for quantifying the balance of a tree, with few of them being directly comparable.
In a recent paper \cite{lemant_robust_2022}, we proposed a new robust, universal index, $J^1$, for quantifying the
balance of rooted trees with arbitrary node degree and size distributions. This index is based on the
Shannon entropy and favours even distributions of node sizes. We showed that $J^1$ is robust, in the sense
that it is insensitive to small changes in node sizes and to the removal of small nodes (include figures
you generated for the paper). We further showed that this index unites and generalises two of the most
popular prior approaches to quantifying tree balance in biology, the Colless index and the Sackin index.
Applied to evolutionary trees, $J^1$ outperforms conventional tree balance indices as a summary statistic
for comparing model output to empirical data \cite{noble_spatial_2022}.\par
Given any tree shape index, an important task is to obtain its expected and extreme values under standard
tree-generating processes, which can then be used as null-model reference points. \cite{lemant_robust_2022}
obtained analytical approximations to the expected values of $J^1$ under the Yule process and the uniform
model, and tested their accuracy numerically for trees with up to $128$ leaves (include figure). In the
same study, we proved that caterpillar trees minimise $J^1$ among bifurcating trees but not when larger
outdegrees are permitted.\par
In chapter \ref{chapter:trees}, we will expand upon three points. First, we further establish $J^1$ as a
universal index of tree balance by identifying funamental connections to classical results in
computer science, related to Huffman coding and self-balancing tree data structures. Second, we derive
upper bounds on the error of the expected value approximations for the Yule process and the uniform models.
For the Yule process, we prove that the approximation rapidly convergest to the true expectation in the
large tree limit. Finally, we investigate the minimal values of $J^1$ in important special cases,
obtaining a counter intuitive result in the large tree limit.

\section{Agent-based modelling in oncology}

\section{Approximate Bayesian computation}

\section{Fluctuating methylation clocks}
