\chapter*{Abstract}
\addcontentsline{toc}{chapter}{Abstract}

Determining the mode of tumour evolution is a fundamental question in cancer
biology. Knowledge of the evolutionary dynamics of tumours could lead to
improved diagnosis and treatment through the identification of key patterns in
the data. In this thesis, I present a multi-pronged approach to the study of
tumour evolution by further developing methods for the analysis of phylogenetic
trees, investigating how different measures of tree properties evolve with
time, and narrowing down the consideration of evolutionary models to those that
are most relevant in colorectal cancer. \par

A recently introduced tree balance index, $J^1$, unlike prior definitions,
permits meaningful comparison of trees with arbitrary outdegree distributions
and node sizes, thus overcoming the shortcomings of conventional methods. I
quantify the accuracy of approximations to the expected values of $J^1$ for two
important null models: the Yule process and the uniform model, and prove that,
for the Yule process, the approximation converges to the true expectation in
the limit of large trees. I further investigate the minima of $J^1$ for
certain important tree families. These results help establish $J^1$ as a
universal, cross-disciplinary index of tree balance that generalizes and
supersedes prior approaches.\par

As balance is only one of several properties that can be used to characterise
phylogenetic trees, I also investigate the evolution of other metrics used in
the study of phylogenies. By recapitulating the results of a previous study with
a slightly altered methodology, and by expanding the analysis to include a new,
more comprehensive set of tree indices, I discuss how these methods could be
used to examine the evolutionary dynamics of tumours. \par

Finally, I develop an agent-based model of colorectal cancer evolution which is
informed by multi-site DNA methylation data. I use this model to infer
properties of mutliple tumours and draw conclusions about the rate of tumour
growth and strength of selection acting on the tumour. I find that the model
is able to reproduce the observed data but not detailed enough to infer the
strength of selection within tumour glands. \par
