\chapter{Agent-based workflow for inferring evolutionary parameters from molecular data using approximate Bayesian computation}\label{demonchap}
In this chapter, I will go over the methods I have used and developed for work on data in chapter \ref{methchap}. 

\section{Introduction}
\subsection{Spatial agent-based modelling}
\begin{itemize}
    \item go over a few relevant models in the field and how they compare to demon
    \item discuss the general assumptions and limitations of ABM
\end{itemize}


\subsection{Approximate Bayesian computation (ABC)}
\begin{itemize}
    \item high-level introduction of ABC
    \item papers in the field which have used some form of ABC
\end{itemize}

\section{Initial simulation workflows}
\begin{itemize}
    \item go over the old \texttt{demon} simulations with \texttt{demonmeth} R package analysis
    \item discuss why the approach worked
    \item point out the ways in which it didn't exactly work (i.e. impossible to get independent methylation and demethylation rates; output files sometimes too large to import into R and analyse efficiently; sometimes large files may not contain all the required data)
\end{itemize}

\section{Simulating fluctuating methylation arrays with \texttt{methdemon}}
\subsection{Overview}
\begin{itemize}
    \item go over the simulation's inner workings
    \item provide estimates of running efficiency and memory requirements
    \item discuss possible upgrades and their potential computational costs
\end{itemize}
\subsection{Examples}
Provide example outputs (and their visualisations), parameter tables and a citation/link to the github repo.

\section{Fluctuating methylation arrays through the lens of ABC}
\subsection{Overview}
\begin{itemize}
    \item go over the \texttt{pyabc} package briefly (cite)
    \item explain the ABC workflow
    \item discuss computational costs and efficiency
    \item discuss whether this is the best approach (can we write down a likelihood for the problem?)
\end{itemize}
\subsection{Examples}
Provide example applications of the workflow to \texttt{methdemon} outputs - fit smaller simulations to a big one for example.
