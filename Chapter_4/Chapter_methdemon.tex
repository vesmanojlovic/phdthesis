\chapter{Agent-based workflow for inferring evolutionary parameters from molecular data using approximate Bayesian computation}\label{chapter:methdemon}


\section{Introduction}
In chapter \ref{chapter:trajectories}, I used a general agent-based model to investigate broad evolutionary
patterns as related to spatial organisation. While the model was capable of simulating the dynamics of
tumour growth, its utility is limited by the computational cost of simulating a large number of cells. This means
that using the model's outputs in comparison to or to draw inference from real data is not feasible. \par
There are a few ways to address this issue. For example, rather than simulating all clones in a tumour, one
could take the approach of \cite{sottoriva_big_2015} and use demes (tumour glands) as the principal agent of our
simulation. This would allow for a realistically-sized tumour to be generated as the number of glands would
be around the right order of magnitude. A problem with this approach is that it loses resolution since a
gland's population is assumed to be clonal, undergoing rapid fixation in the case of an emerging mutant. If we wanted to
study evolutionary dynamics on a finer scale we would need to at least simulate the dynamics of cell
lineages, if not individual cells, as performed earlier. \par
In \cite{gabbutt_evolutionary_2023}, the authors employ a stochastic model for an expanding cell population
to model the behaviour of fluctuating CpG sites in blood cancers. The model is capable of simulating the
dynamics and the corresponding fluctuating methylation arrays of lymphoid malignancies at scale. However,
this model is not spatially explicit, which is a feature we need to be able to discriminate between different
glands in a solid tumour. In this chapter, I present a purpose-written agent-based model, \texttt{methdemon},
which reduces the computational cost of simulating a tumour's growth and models the fluctuating
methylation arrays in colorectal cancer.


\section{Background on colorectal cancer --- model assumptions}

\subsection{Colorectal cancer evolution}\label{section:assumptions:general}

The most common type of colorectal cancer is adenocarcinoma, which arises from the epithelial cells lining
the colon, covering more than $90\%$ of cases. Most diagnosed
colorectal cancers are moderately to well differentiated, meaning that at least $50\%$ of the tumour's cells
form glands \cite{fleming_colorectal_2012}.
As it is a solid tumour, one cannot ignore the spatial organisation of the cells. Its
origin, tumorigenesis, follows an accumulation of mutations in a cell's DNA, which leads to the cell's transformation
from a healthy to malignant cell \cite{fearon_genetic_1990}. Once the malignancy is established, the tumour
forms hierarchical cell structures similar to those of normal tissue \cite{cernat_colorectal_2014}, organising
into crypt-like glands. The tumour spreads by the process of gland fission, which happens when a gland's
population grows large enough to split into two glands \cite{preston_bottom-up_2003}. \par
Translated into the language of an agent-based model, we can write down our initial assumptions as follows:
\begin{enumerate}[(i)]
    \item \textbf{A single cell forms the first gland and initiates tumour growth.} This assumption skips
        over the process of tumorigenesis, during which a cell accumulates mutations and becomes
        malignant \cite{tariq_colorectal_2016}. This is a simplification to be sure, but a reasonable one,
        given that the focus of this work is on the evolutionary dynamics of the tumour rather than its
        initiation.
    \item \textbf{The rate of driver mutations is Poisson distributed and identical for all cells.} This
        assumption is consistent with most models of tumour evolution \cite{niida_modeling_2021}.
    \item \textbf{The cell population within a gland grows exponentially and is well-mixed.} While not necessarily
        consistent with the biology of a solid tumour, this assumption allows for more efficiency in the simulation
        as opposed to a multi-level spatial model. Further, as the data discussed in chapter \ref{chapter:methylation}
        is obtained from bulk samples of tumour glands, this assumption is not unreasonable. \label{item:well_mixed}
    \item \textbf{Once a gland reaches a certain size, which we call the carrying capacity, the population undergoes
        steady-state turnover according to the Moran process.}
    \item \textbf{At carrying capacity, a gland has a certain probability of undergoing fission, which splits the
        gland's population randomly into two.} As a consequence of assumption (\ref{item:well_mixed}), fissions do not
        take into account a gland's spatial organisation.
    \item \textbf{Gland fission occurs as a neutral spatial branching process.} The previous two assumptions and this
        one together form the basis of the model's spatial dynamics. While there are other mechanisms of colorectal
        adenocarcinoma progression, gland fission is the principal way in which the tumour grows \cite{preston_bottom-up_2003}.
        The assumption of neutrality in the spatial branching process is consistent with the findings of \cite{sottoriva_big_2015}.
\end{enumerate}

\subsection{Fluctuating methylation arrays}
To my shock, much like many mathematicians before me, the availability of perfectly clean data containing detailed information
about the population structure of a tumour is non-existent. This is mainly because it is impossible to obtain it with current
technology. What one learns very quickly when working with biological data is to compromise. Specifically, when it comes to
cancer, a compromise has to be made between resolution and scale. Where single-cell data can provide a detailed view of the
mutations accumulated in the genome, it is not feasible to obtain it for a whole tumour. On the other hand, bulk data
gives a high-level view of the tumour's population structure, but a lot of the details get lost in the process. \par
However, DNA sequencing is not the only way to obtain information about a tumour's population structure. Early work with
methylation arrays in colorectal cancer has shown potential for inferring the ancestry and age of a tumour \cite{hong_using_2010,
siegmund_high_2011}. In a way the genome shows more mutations in older populations, methylation arrays will also be more diverse
as time goes on. Current techniques allow for the sequencing of some $850,000$ CpG sites which, while a small fraction of the
genome, is still enough to proide valuable insight into the underlying dynamics of the cell population. Initial studies on
methylation as a tracker of evolutionmade use of the whole array \cite{siegmund_modeling_2008, sottoriva_integrating_2010}.
However, more recent work has shown that just a small subset of CpG sites is enough to infer the evolutionary dynamics of a
cell population \cite{gabbutt_fluctuating_2022, gabbutt_evolutionary_2023}. This is the set of fluctuating CpG (fCpG) loci,
which is also the topics of chapter \ref{chapter:methylation}. \par
On top of the assumptions outlined in section \ref{section:assumptions:general}, we include the following set for the modelling
of fluctuating methylation arrays:
\begin{enumerate}[(i')]
    \item \textbf{Each cell has a corresponding fCpG array inherited from its parent cell.}
    \item \textbf{Upon cell division, each methylated fCpG site has an equal and independent probability of being demethylated,
        and vice-versa.} This assumption is based on the findings of \cite{gabbutt_fluctuating_2022, gabbutt_evolutionary_2023}.
    \item \textbf{The rates of methylation and demethylation do not change over time.}
\end{enumerate}


\section{Existing simulation workflows}\label{section:old_famework}

With the assumptions outlined in the previous section, my initial approach was to employ a general agent-based model
with small modifications, due to the specific data type but broad evolutionary questions.
\textbf{TO DO: discuss the shortcomings of the old approach.}

\begin{figure}[h]
    \centering
    \includegraphics[width=0.8\textwidth]{Chapter_4/figures/old_workflow1.png}
    \caption{The old workflow for simulating tumour growth and assigning methylation arrays.}
    \label{fig:old_workflow1}
\end{figure}
\begin{figure}[h]
    \centering
    \includegraphics[width=0.8\textwidth]{Chapter_4/figures/old_workflow2.png}
    \caption{The old workflow for simulating tumour growth and assigning methylation arrays.}
    \label{fig:old_workflow2}
\end{figure}

While informative on a high level, this approach had multiple issues. Firstly, the simulations were
limited to a couple million cells which, after assigning across glands, resulted in a tumour too small
to reasonably compare to real data. Secondly, previous work on fluctuating methylation arrays
\cite{gabbutt_fluctuating_2022, gabbutt_evolutionary_2023} has shown that the methylation and demethylation
rates are not necessarily equal, which is not a feature of the \texttt{demon} simulations, where we used
passenger mutations as a proxy for epigenetic changes. However, these issues may not have been insurmountable
with clever tweaks to code. The final nail in the coffin came with the size of output files, as they were often
unwieldy and difficult to handle by the R compiler. This meant that we would have to alter our approach in a way
which allows for a leaner output, while keeping the spatial aspect of the evolutionary dynamics of the tumour intact.
For that purpose, we developed a new agent-based model, \texttt{methdemon}.


\section{Simulating fluctuating methylation arrays}\label{section:methdemon}
\subsection{Overview}
\begin{itemize}
    \item go over the simulation's inner workings
    \item provide estimates of running efficiency and memory requirements
    \item discuss possible upgrades and their potential computational costs
\end{itemize}
\subsection{Examples}
Provide example outputs (and their visualisations), parameter tables and a citation/link to the github repo.

\section{Fluctuating methylation arrays through the lens of ABC}\label{section:methabc}
\subsection{Overview}
\begin{itemize}
    \item go over the \texttt{pyabc} package briefly (cite)
    \item explain the ABC workflow
    \item discuss computational costs and efficiency
    \item discuss whether this is the best approach (can we write down a likelihood for the problem?)
\end{itemize}
\subsection{Examples}
Provide example applications of the workflow to \texttt{methdemon} outputs - fit smaller simulations to a big one for example.
