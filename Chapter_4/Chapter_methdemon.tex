\chapter{Agent-based workflow for inferring evolutionary parameters from molecular data using approximate Bayesian computation}\label{chapter:methdemon}


\section{Introduction}
In chapter \ref{chapter:trajectories}, we used a general agent-based model to investigate broad evolutionary
patterns as related to spatial organisation. While the model was capable of simulating the dynamics of
tumour growth, its utility is limited by the computational cost of going beyond $10^6$ cells. This means
that using the model's outputs in comparison to or to draw inference from real data is not feasible. \par
There are a few ways to address this issue. For example, rather than simulating all clones in a tumour, we
could take the approach of \cite{sottoriva_big_2015} and use demes (tumour glands) as the agent of our
simulation. This would allow for a realistically-sized tumour to be generated as the number of glands would
be around the right order of magnitude. A problem with this approach is that it loses resolution since a
gland's population is assumed to have undergone a sweep the moment a mutation arises. If we wanted to
study evolutionary dynamics on a finer scale we would need to at least simulate the dynamics of cell
lineages, if not individual cells. \par
In \cite{gabbutt_evolutionary_2023}, the authors employ a stochastic model for an expanding cell population
to model the behaviour of fluctuating CpG sites in blood cancers. The model is capable of simulating the
dynamics and the corresponding fluctuating methylation arrays of lymphoid malignancies at scale. However,
this model does not suit our purposes either as it is not spatially explicit, and we need to be able to
discriminate between tumour glands in colorectal cancer based on their spatial organisation.


\section{Colorectal adenocarcinoma}
Colorectal adenocarcinoma is a type of cancer that arises from the epithelial cells lining the colon.
It is a solid tumour, which means we cannot ignore the spatial organisation of the cells. The tumour's
origin, tumorigenesis, follows an accumulation of mutations in a cell's DNA, which leads to the cell's transformation
from a healthy to malignant cell \cite{fearon_genetic_1990}. Once the malignancy is established, the tumour
forms hierarchical cell structures similar to those of normal tissue \cite{cernat_colorectal_2014}, organising
into crypt-like glands. The tumour spread by the process of gland fission, which happens when a gland's
population grows large enough to split into two glands \cite{preston_bottom-up_2003}. This process is what
informs our apprach to simulating the tumour's growth.


\section{Modifying existing simulation workflows}\label{section:old_famework}
% \begin{itemize}
%     \item go over the old \texttt{demon} simulations with \texttt{demonmeth} R package analysis
%     \item discuss why the approach worked
%     \item point out the ways in which it didn't exactly work (i.e. impossible to get independent methylation and demethylation rates; output files sometimes too large to import into R and analyse efficiently; sometimes large files may not contain all the required data)
% \end{itemize}
An analytical approach to modelling gland fission is difficult, due to the complexity of the process,
as each gland has its own internal evolutionray dynamics in addition to spatial organisation in relation
to other glands. Furthermore, fission splits a gland's population into two, with the daughter glands'
spatial organisation further complicating the model. The issue arises when we consider the possibilities
of boundary growth versus gland fissions being allowed throughout the tumour. If we limit the discussion
to cells rather than glands, boundary growth seems to result in cell phylogenies with longer branches than
unrestricted growth \cite{lewinsohn_state-dependent_2023}, which would make the latter more consistent with
neutral evolution \cite{sottoriva_big_2015, williams_identification_2016}. However, as solid tumours often have spatial restrictions,
one would not expect a purely unrestricted growth model to be realistic. To investigate how these spatial
models affect a tumour's evolutionary dynamics and the corresponding methylation arrays, we used \texttt{demon}
to simulate different scenarios. The output data was then assigned methylation arrays using \texttt{demonmeth},
a proprietary R package (figures \ref{fig:old_workflow1} and \ref{fig:old_workflow2}).

\begin{figure}[h]
    \centering
    \includegraphics[width=0.8\textwidth]{Chapter_4/figures/old_workflow1.png}
    \caption{The old workflow for simulating tumour growth and assigning methylation arrays.}
    \label{fig:old_workflow1}
\end{figure}
\begin{figure}[h]
    \centering
    \includegraphics[width=0.8\textwidth]{Chapter_4/figures/old_workflow2.png}
    \caption{The old workflow for simulating tumour growth and assigning methylation arrays.}
    \label{fig:old_workflow2}
\end{figure}

While informative on a high level, this approach had multiple issues. Firstly, the simulations were
limited to a couple million cells which, after assigning across glands, resulted in a tumour too small
to reasonably compare to real data. Secondly, previous work on fluctuating methylation arrays
\cite{gabbutt_fluctuating_2022, gabbutt_evolutionary_2023} has shown that the methylation and demethylation
rates are not necessarily equal, which is not a feature of the \texttt{demon} simulations, where we used
passenger mutations as a proxy for epigenetic changes. However, these issues may not have been insurmountable
with clever tweaks to code. The final nail in the coffin came with the size of output files, as they were often
unwieldy and difficult to handle by the R compiler. This meant that we would have to alter our approach in a way
which allows for a leaner output, while keeping the spatial aspect of the evolutionary dynamics of the tumour intact.
For that purpose, we developed a new agent-based model, \texttt{methdemon}.


\section{Simulating fluctuating methylation arrays}\label{section:methdemon}
\subsection{Overview}
\begin{itemize}
    \item go over the simulation's inner workings
    \item provide estimates of running efficiency and memory requirements
    \item discuss possible upgrades and their potential computational costs
\end{itemize}
\subsection{Examples}
Provide example outputs (and their visualisations), parameter tables and a citation/link to the github repo.

\section{Fluctuating methylation arrays through the lens of ABC}\label{section:methabc}
\subsection{Overview}
\begin{itemize}
    \item go over the \texttt{pyabc} package briefly (cite)
    \item explain the ABC workflow
    \item discuss computational costs and efficiency
    \item discuss whether this is the best approach (can we write down a likelihood for the problem?)
\end{itemize}
\subsection{Examples}
Provide example applications of the workflow to \texttt{methdemon} outputs - fit smaller simulations to a big one for example.
